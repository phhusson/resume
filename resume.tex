%% start of file `template.tex'.
%% Copyright 2006-2015 Xavier Danaux (xdanaux@gmail.com).
%
% This work may be distributed and/or modified under the
% conditions of the LaTeX Project Public License version 1.3c,
% available at http://www.latex-project.org/lppl/.

\documentclass[12pt,a4paper,sans]{moderncv}        % possible options include font size ('10pt', '11pt' and '12pt'), paper size ('a4paper', 'letterpaper', 'a5paper', 'legalpaper', 'executivepaper' and 'landscape') and font family ('sans' and 'roman')

% moderncv themes
\moderncvstyle{classic}                             % style options are 'casual' (default), 'classic', 'banking', 'oldstyle' and 'fancy'
\moderncvcolor{blue}                               % color options 'black', 'blue' (default), 'burgundy', 'green', 'grey', 'orange', 'purple' and 'red'
%\renewcommand{\familydefault}{\sfdefault}         % to set the default font; use '\sfdefault' for the default sans serif font, '\rmdefault' for the default roman one, or any tex font name
%\nopagenumbers{}                                  % uncomment to suppress automatic page numbering for CVs longer than one page

% character encoding
%\usepackage[utf8]{inputenc}                       % if you are not using xelatex ou lualatex, replace by the encoding you are using
%\usepackage{CJKutf8}                              % if you need to use CJK to typeset your resume in Chinese, Japanese or Korean
%\usepackage{hyperref}

% adjust the page margins
\usepackage[margin=0.5in,scale=1.0]{geometry}
%\setlength{\hintscolumnwidth}{3cm}                % if you want to change the width of the column with the dates
%\setlength{\makecvtitlenamewidth}{10cm}           % for the 'classic' style, if you want to force the width allocated to your name and avoid line breaks. be careful though, the length is normally calculated to avoid any overlap with your personal info; use this at your own typographical risks...

% personal data
\name{Pierre-Hugues}{Husson}
%\title{Resumé title}                               % optional, remove / comment the line if not wanted
%\address{street and number}{postcode city}{country}% optional, remove / comment the line if not wanted; the "postcode city" and "country" arguments can be omitted or provided empty
\phone[mobile]{+33~6~06~65~67~06}                   % optional, remove / comment the line if not wanted; the optional "type" of the phone can be "mobile" (default), "fixed" or "fax"
\email{phh@phh.me}                               % optional, remove / comment the line if not wanted
%\homepage{www.johndoe.com}                         % optional, remove / comment the line if not wanted
%\social[linkedin]{john.doe}                        % optional, remove / comment the line if not wanted
\social[twitter]{phhusson}                             % optional, remove / comment the line if not wanted
\social[github]{phhusson}                              % optional, remove / comment the line if not wanted
%\extrainfo{additional information}                 % optional, remove / comment the line if not wanted
%\photo[64pt][0.4pt]{picture}                       % optional, remove / comment the line if not wanted; '64pt' is the height the picture must be resized to, 0.4pt is the thickness of the frame around it (put it to 0pt for no frame) and 'picture' is the name of the picture file
%\quote{Some quote}                                 % optional, remove / comment the line if not wanted

% bibliography adjustements (only useful if you make citations in your resume, or print a list of publications using BibTeX)
%   to show numerical labels in the bibliography (default is to show no labels)
\makeatletter\renewcommand*{\bibliographyitemlabel}{\@biblabel{\arabic{enumiv}}}\makeatother
%   to redefine the bibliography heading string ("Publications")
%\renewcommand{\refname}{Articles}

% bibliography with mutiple entries
%\usepackage{multibib}
%\newcites{book,misc}{{Books},{Others}}
%----------------------------------------------------------------------------------
%            content
%----------------------------------------------------------------------------------

\begin{document}
%\begin{CJK*}{UTF8}{gbsn}                          % to typeset your resume in Chinese using CJK
%-----       resume       ---------------------------------------------------------
%	\vspace*{-15mm}
\makecvtitle
%	\vspace*{-10mm}

\vspace*{-3mm}
\section{Work experience}
\cventry{2020--}{Staff Software engineer}{Freebox}{Paris}{}{
	\begin{itemize}
        \item Lead a team of two
        \item Responsible for long-term maintainance \& deployment (security, partners, Google cert)
        \item Lots of bug fixes. Areas:
            \begin{itemize}
                \item Always-on micro-controller (power consumption, alarms)
                \item Rewrote Audio HAL from scratch
                \item Android java frameworks
                \item Kernel drivers (CEC, WiFi, HDR, AVB, Ethernet, i2c, HDCP...)
                \item Bootloader (security, refurbishing)
            \end{itemize}
        \item Migrated from ODM-led work-tree, to Freebox-led work-tree
        \item Exoplayer implementation of proprietary RTSP with Widevine DRM
        \item Re-wrote a Widevine Trusted Application
        \item Software evaluation of STB SoC vendors
        \item PoC Ported Android 14 and Linux 6.6 on Broadcom Android 7/Linux 3.14 drivers
        \item Low-RAM optimizations for Android 7 platform
        \item Implemented IR learning for remote control
        \item Project management with ODM (SEI Robotics)
        \item Re-wrote a FTA DVB-T TVInputService with ExoPlayer using Linux-DVB driver
        \item Linux mainline contributions to better support our products
        \item Collect anonymous statistics (backend, on-device collector and dashboard)
        \item RMA-specific software
        \item Prototype LLM-based voice assistant
            \begin{itemize}
                \item End-to-end PoC
                \item Evaluation of ASR providers
                \item Experimentations with RAG (text \& semantic search database)
                \item Tool calling \& MCP
                \item PoC backend (python, flask)
                \item Integration with legacy Nuance-based protocol
                \item Conversational near-field (modification of RCU for conversational)
                \item Voice UX mock-ups (audio fillers, interruptions...)
            \end{itemize}
        \item \href{https://github.com/phhusson/speechseparation}{AI model for speech improvement on TV broadcasts}
            \begin{itemize}
                \item Study the state of the art research
                \item Integration of 3P catalogues
                \item Discussing requirements with 3P
                \item Adapt model to run on 1 core Cortex-A53 (Target: 2019 STB, no NPU, poor GPU)
                \item Integration into Android audio flinger
                \item UI implementation, following product team's guidelines
            \end{itemize}
	\end{itemize}
}
\cventry{2018--2020}{Software engineer, kickstarting Android}{Softathome}{Colombes, Paris Area}{}{
	\begin{itemize}
        \item Android firmware/apps CI
        \item Android TV Generic System Image PoC (8 SoCs vendors)
            \begin{itemize}
                \item 8 SoC vendors (STBs, tablets and smartphones for all-in-one demos)
                \item Demo of Android 9 and Android 10, 2 weeks after their source code release
            \end{itemize}
        \item Android-stack tv player (ExoPlayer-based)
            \begin{itemize}
                \item Partial HbbTV support (\href{https://github.com/google/ExoPlayer/pull/6922}{upstream part})
                \item \href{https://github.com/google/ExoPlayer/pull/7005}{Improved SmoothStreaming support}
                \item Viaccess DRM integration
                \item DVB/multicast support (DVB-T, DVB-SI, CAS...)
                \item Seamless ad-insertion prototype
                \item Faster time-to-zap optimizations (down to 120ms on multicast IPTV, 200ms on OTT)
            \end{itemize}
        \item Android metrics analysis
        \item Development-focused userspace bootloader based on kexec
            \begin{itemize}
                \item Automatically boots either latest revision or CI-controlled revision
                \item Boot over network
            \end{itemize}
        \item CI for Softathome's Linux-based STB software solution
            \begin{itemize}
                \item Optimization of build time (2 hours => 13 minutes)
                \item Integration of Gitlab and Jenkins into Softathome's build system
                \item Automatic integration test run on STBs at every merge-request
            \end{itemize}
         \item Android platforms security analysis
            \begin{itemize}
                \item Reported Amlogic STB security flaw ranked High in Android security bulletin
                \item Reported two Broadcom STB security flaws ranked High
                \item Discussions with DRM/CAS provider to help them improve their security, and Android compatibility
            \end{itemize}
         \item Rust implementation of RTP multicast receiver, including FEC and FCC support
	\end{itemize}
}
\cventry{2016--2017}{Lead software engineer}{Archos}{Igny, Paris Area}{}{
	\begin{itemize}
		\item Manage B2B customer technical relationship and development
		\item Drive and train engineer team in Archos Shenzhen
		\item Manage a team of 4 engineers in France
		\item Manage remotely a team of 4 in Shenzhen
        \item Handle the release, including Google certification, of nearly one new model per week
	\end{itemize}
}
\cventry{2014--2015}{Software engineer}{Archos}{Igny, Paris Area}{}{
	\begin{itemize}
		\item Certification Android devices for Google services (CTS/GTS)
		\item Creation of Android firmware-editing tools
		\item Creation of Archos Fusion Storage
		\item Development and operation of Archos OTA system
		\item Collection and analysis of device statistics (based on ES/Kibana)
        \item Development of testing tools and CI (Android firmwares, CTS, GTS, apps, ...)
			\begin{itemize}
				\item Automatic Android firmware tests
				\item CTS/GTS integration
				\item List and fixes of known usual problems
				\item Continuous build of various Android apps
			\end{itemize}
		\item Support of various SoC vendors (TI, Rockchip, Qualcomm, Unisoc, MTK, ...)
		\item Early-detection and removal of in-firmware PHAs
		\item \href{https://github.com/archos-sa/security-binary}{Binary security patches experiment}
		\item Contributions to Archos Video Player
			\begin{itemize}
				\item Audio passthrough support
				\item Torrent streaming support
			\end{itemize}
	\end{itemize}}
\cventry{2013}{Intern}{SeQureNet}{Paris}{}{
	\begin{itemize}
		\item PCBA conception (schematics, routing)
		\item FPGA programming
		\item Linux user-space driver for FPGA
	\end{itemize}
}

\newpage

\section{Education}
\cventry{2010--2013}{Master Degree in Engineering}{Ing\'enieur Telecom ParisTech}{}{}{Major: Embedded systems}

%\vspace*{-3mm}
\section{Voluntary experience}
\cventry{2018--}{\href{https://github.com/nova-video-player/aos-AVP}{Nova Video Player}}{full-featured Android Video Player, including library}{}{}{
    \begin{itemize}
        \item CI
        \item FDroid integration
        \item Torrent support maintainance
        \item SFTP stability
        \item Full-text search prototypes
    \end{itemize}
}
\cventry{2022--}{\href{https://github.com/TrebleDroid/treble_experimentations/wiki}{TrebleDroid}}{Android Generic System Image}{}{}{
    \begin{itemize}
        \item Continuation of Phh-Treble as a team
        \item Leading \& maintaining
    \end{itemize}
}
\cventry{2017--2022}{\href{https://github.com/phhusson/treble_experimentations/wiki}{Phh-Treble}}{Android Generic System Image}{}{}{
    \begin{itemize}
        \item Maintaining support for devices originally running Android 8.0 or more recent
        \item Vendor-specific workarounds in Android framework
        \item Support for vendor-specific features
        \item OTA mechanism based on Android 10's dynamic partitions
        \item Used as technical basis for 10+ other custom GSIs, including Ubuntu Touch
        \item Regular discussions with Google to discuss future of Project Treble
        \item Contributions reused across the whole custom ROM community, including LineageOS
    \end{itemize}
}
%\cventry{2013}{Author and maintainer}{quassel-irssi}{}{}{Creation of an IM module for irssi, to connect to a quassel core}
\cventry{2015--2017}{\href{https://forum.xda-developers.com/android/software-hacking/wip-selinux-capable-superuser-t3216394}{phh's SuperUser}}{An SELinux-aware Android super-user}{}{}{}
\cventry{2010--2015}{Rezel voluntary association}{Telecom ParisTech student house}{}{}{
	President of the association
	\newline{}
	Managing and maintaining the local network (circa. 500 subscribers)
}
%\cventry{2010--2013}{Telecom robotics}{Telecom ParisTech student robotics association}{}{}{
%	\begin{itemize}
%		\item Creation of a C++11 framework for STM32F4 micro-controller
%		\item Use of Ada programming language
%		\item Basic use of vision detection
%	\end{itemize}
%}
\cventry{2008--2010}{\href{https://forum.xda-developers.com/showthread.php?t=601751}{XDandroid}}{Android port to Windows CE-based phones}{}{}{}
%	\begin{itemize}
%		\item \href{Fork of existing https://github.com/phhusson/Superuser}{SU app}
%		\item \href{Creation of an https://github.com/phhusson/sepolicy-inject}{SELinux policy edition tool}
%		\item \href{Creation of a https://github.com/phhusson/super-bootimg}{boot.img file format editor}
%	\end{itemize}
%}
%\cventry{2013--2014}{STpp}{\href{https://github.com/phhusson/STpp}{C++ framework for STM32F4 microcontrollers}}{}{}{}
\cventry{2010--2011}{Toshiba AC100 opensource support}{ARM-based laptop}{}{}{
	\begin{itemize}
		\item NVEC reverse engineering and driver creation (driver mainlined since then)
		\item Userspace audio codec driver
	\end{itemize}
}
\cventry{}{Miscellaneous open-source contributions}{}{}{}{}


%\cvlistdoubleitem{TWRP support for some Archos devices}{\href{https://github.com/lkl/linux/commits?author=phhusson}{Bugfixes in Linux Kernel Library}}
\cvlistdoubleitem{\href{https://patchwork.kernel.org/patch/9413747/}{Linux support of RC5T619 PMIC}}{\href{https://github.com/iBotPeaches/Apktool/commits?author=phhusson}{Bugfixes in Apktool}}
\cvlistitem{\href{https://github.com/rockchip-linux/rockchip_forwardports}{Rockchip ports (mali driver, vtl\_ts, build fixes, codec, ...)}}
\cvlistitem{\href{https://forum.xda-developers.com/android/software/wifi-dhcp-accelerator-t3023534}{DHCP on Android connection-time analysis}}



%	\begin{itemize}
%		\item %		\item %		\item %	\end{itemize}
%}

%\section{Languages}
%\cvitem{French}{Native speaker}
%\cvitem{English}{Fluent}

%\section{Computer skills}
%\cvitem{Programming languages}{Java, C, C++, Scala, shell, javascript, ...}
%\cvitem{VCS}{git, mercurial, subversion, ...}

% Publications from a BibTeX file without multibib
%  for numerical labels: \renewcommand{\bibliographyitemlabel}{\@biblabel{\arabic{enumiv}}}% CONSIDER MERGING WITH PREAMBLE PART
%  to redefine the heading string ("Publications"): \renewcommand{\refname}{Articles}
\nocite{*}
\bibliographystyle{plain}
\bibliography{publications}                        % 'publications' is the name of a BibTeX file

% Publications from a BibTeX file using the multibib package
%\section{Publications}
%\nocitebook{book1,book2}
%\bibliographystylebook{plain}
%\bibliographybook{publications}                   % 'publications' is the name of a BibTeX file
%\nocitemisc{misc1,misc2,misc3}
%\bibliographystylemisc{plain}
%\bibliographymisc{publications}                   % 'publications' is the name of a BibTeX file

%\clearpage\end{CJK*}                              % if you are typesetting your resume in Chinese using CJK; the \clearpage is required for fancyhdr to work correctly with CJK, though it kills the page numbering by making \lastpage undefined
\end{document}


%% end of file `template.tex'.
